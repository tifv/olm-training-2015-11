% $date: 2015-11-12
% $timetable:
%   g9r2:
%     2015-11-12:
%       2:
%   g9r1:
%     2015-11-12:
%       1:

\section*{Вводное подобие}

% $authors:
% - Фёдор Львович Бахарев

\begin{problems}

\item
Дана точка~$A$ и~окружность~$\omega$, не~проходящая через $A$.
Докажите, что
\\
\subproblem
все окружности, проходящие через точку~$A$ и~пересекающие окружность~$\omega$
в~диаметрально противоположных точках, проходят одновременно через некоторую
точку~$A'$, отличную от~$A$;
\\
\subproblem
все окружности, проходящие через точку~$A$ и~перпендикулярные
к~окружности~$\omega$, проходят одновременно через некоторую точку~$A'$,
отличную от~$A$.

\item
Четырехугольник $ABCD$ вписан в~окружность.
Докажите, что касательные к~окружности в~точках $B$ и~$D$ пересекаются
на~прямой~$AC$ тогда и~только тогда, когда $AB \cdot CD = AD \cdot BC$.

\item
Четырехугольник $ABCD$ вписан в~окружность, $M$~--- середина диагонали~$AC$.
Докажите, что $\angle AMB = \angle ADC$ тогда и~только тогда, когда
$AB \cdot CD = AD \cdot BC$.

\item
Четырехугольник $ABCD$ вписан в~окружность, $M$~--- середина диагонали~$AC$.
Докажите, что $\angle AMB = \angle AMD$ тогда и~только тогда, когда
$AB \cdot CD = AD \cdot BC$.

\item
В~параллелограмме $ABCD$ дана точка~$M$, такая что $\angle MAD = \angle MCD$.
Докажите, что $\angle MBA = \angle MDA$.

\item
На~отрезке~$MN$ построены подобные, одинаково ориентированные треугольники
$AMN$, $NBM$ и~$MNC$.
Докажите, что треугольник $ABC$ подобен всем этим треугольникам, а~центр его
описанной окружности равноудален от~точек $M$ и~$N$.

\item
На~сторонах $AB$, $BC$ и~$CA$ треугольника $ABC$ зеленой краской отметили
соответственно точки $C_1$, $A_1$ и~$B_1$, отличные от~вершин треугольника.
Оказалось, что
\(
    A C_1 : C_1 B
=
    B A_1 : A_1 C
=
    C B_1 : B_1 A
\),
%\[
%    \frac{A C_1}{C_1 B}
%=
%    \frac{B A_1}{A_1 C}
%=
%    \frac{C B_1}{B_1 A}
%\; , \]
а~$\angle BAC = \angle B_1 A_1 C_1$.
Докажите, что треугольник с~зелеными вершинами подобен треугольнику $ABC$.

\item
Окружность, проходящая через вершины $A$ и~$B$ треугольника $ABC$, пересекает
сторону~$BC$ в~точке~$D$.
Окружность, проходящая через вершины $B$ и~$C$, пересекает сторону~$AB$
в~точке~$E$ и~первую окружность вторично в~точке~$F$.
Оказалось, что точки $A$, $E$, $D$, $C$ лежат на~окружности с~центром~$O$.
Докажите, что угол $BFO$~--- прямой.

\end{problems}

