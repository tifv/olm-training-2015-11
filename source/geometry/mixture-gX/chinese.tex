% $date: 2015-11-15
% $timetable:
%   gX:
%     2015-11-15:
%       2:

\section*{Китайская геометрия}

% $authors:
% - Фёдор Львович Бахарев

\begin{problems}

\item
Окружность~$\Gamma$ проходит через вершину~$A$ треугольника $ABC$ и~пересекает
стороны $AB$ и~$AC$ в~точках $E$ и~$F$ соответственно, а~описанную окружность
треугольника $ABC$ вторично в~точке~$P$.
Докажите, что точка, симметричная $P$ относительно $EF$, лежит на~$BC$ тогда
и~только тогда, когда $\Gamma$ проходит через центр описанной окружности
треугольника $ABC$.

\item
В~остроугольном треугольнике $ABC$ точка~$O$~--- центр описанной окружности,
$G$~--- центр масс.
Пусть $D$~--- середина стороны~$BC$, $E$~--- точка на~окружности
с~диаметром~$BC$, лежащая внутри треугольника $ABC$, такая что $AE \perp BC$.
Пусть $F = EG \cap OD$ и~точки $K$ и~$L$ на~прямой~$BC$ таковы, что
$FK \parallel OB$ и~$FL \parallel OC$.
Кроме того, точка $M \in AB$ такова, что $MK \perp BC$ и~точка $N \in AC$
такова, что $NL \perp BC$.
Наконец, окружность~$\omega$ касается отрезков $OB$ и~$OC$ в~точках $B$ и~$C$.
Если вы дочитали условие до~конца, то~докажите, что описанная окружность
треугольника $AMN$ касается окружности~$\omega$.

\item
Внутри треугольника $ABC$ дана точка~$P$.
Предположим, что $L$, $M$ и~$N$~--- середины сторон $BC$, $CA$ и~$AB$
соответственно и~$PL : PM : PN = BC : CA : AB$.
Продолжения $AP$, $BP$ и~$CP$ пересекают описанную окружность треугольника
$ABC$ в~точках $D$, $E$ и~$F$ соответственно.
Докажите, что центры описанных окружностей треугольников
$APF$, $APE$, $BPF$, $BPD$, $CPD$, $CPE$
лежат на~одной окружности.

\item
Вписанная окружность неравнобедренного треугольника $ABC$ касается его
сторон $BC$, $AC$, $AB$ в~точках $D$, $E$ и~$F$ соответственно.
Пусть точки $L$, $M$, $N$ симметричны точкам $D$, $E$ и~$F$ соответственно
относительно $EF$, $FD$ и~$DE$.
Пусть $AL \cap BC = P$, $BM \cap CA = Q$ и~$CN \cap AB = R$.
Докажите, что точки $P$, $Q$ и~$R$ коллинеарны.

\item
В~остроугольном треугольнике $ABC$ точки $M$ и~$N$~--- середины
сторон $AB$ и~$AC$ соответственно, $AD$~--- высота.
Окружности $\Gamma_1$ и~$\Gamma_2$ описаны около треугольников $BDM$ и~$CDN$
соответственно и~повторно пересекаются в~точке~$K$.
Точка~$P$ лежит на~$BC$, а~точки $E$ и~$F$ на~$AC$ и~$AB$ соответственно так,
что $PEAF$~--- параллелограмм.
Докажите, что если $MN$~--- общая касательная для $\Gamma_1$ и~$\Gamma_2$,
то~$K$, $E$, $A$, $F$ лежат на~одной окружности.

\item
В~остроугольном треугольнике $ABC$ выполнено соотношение $AB > AC$.
Точка~$M$~--- середина $BC$.
Точка~$P$ внутри треугольника $AMC$ такова, что $\angle MAB = \angle PAC$.
Пусть $O$, $O_1$, $O_2$~--- центры описанных окружностей
треугольников $ABC$, $ABP$ и~$ACP$ соответственно.
Докажите, что $AO$ проходит через середину $O_1 O_2$.

\end{problems}

