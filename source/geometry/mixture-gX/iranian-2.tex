% $date: 2015-11-14
% $timetable:
%   gX:
%     2015-11-14:
%       1:

\section*{Иранская геометрия --- 2}

% $authors:
% - Фёдор Львович Бахарев

\begin{problems}

\item
Пятиугольник $ABCDE$ вписан в~окружность~$\omega$ с~центром~$O$.
Пусть $BE \cap AD = T$.
Прямая, параллельная $CD$ и проходящая через $T$, пересекает $AB$ и~$CE$
в~точках $X$ и~$Y$ соответственно.
Докажите, что окружность, описанная около треугольника $AXY$,
касается $\omega$.

\item
Треугольник $ABC$ вписан в~окружность~$\omega$ с~центром в~точке~$O$.
Пусть $M$ и~$N$~--- середины дуг $AB$ и~$AC$ соответственно
(не~содержащих точки $C$ и~$B$), а~$M'$ и~$N'$~--- точки касания вписанной
окружности треугольника $ABC$ со~сторонами $AB$ и~$AC$
соответственно.
Предположим, что $X$ и~$Y$~--- основания перпендикуляров из~$A$ на~$MM'$
и~$NN'$.
Докажите, что если $I$~--- центр вписанной окружности,
то~четырехугольник $AXIY$ вписаный тогда и~только тогда, когда
$CA + AB = 2 BC$.

\item
Точки $A$, $B$, $C$ и~$D$ лежат на~прямой~$\ell$.
Окружности $\omega_1$ и~$\omega_2$ с~центрами $O_1$ и~$O_2$ проходят через
точки $A$ и~$B$, а~окружности $\omega'_1$ и~$\omega'_2$ с~центрами $O_1'$
и~$O_2'$ проходят через $C$ и~$D$.
Пусть $\omega_1 \perp \omega'_1$ и~$\omega_2 \perp \omega'_2$.
Докажите, что прямые $O_1 O'_2$ , $O_2 O'_1$ и~$\ell$ пересекаются в~одной
точке.

\item
Треугольник $ABC$ вписан в~окружность~$\omega$ с~центром~$O$.
Высота из~точки~$A$ пересекает окружность~$\omega$ в~точке~$D$.
Высоты $BE$ и~$CF$ пересекаются в~ортоцентре~$H$.
$T$~--- середина $AH$.
Прямая, проходящая через $T$ параллельно $EF$, пересекает $AB$ и~$AC$ в~точках
$X$ и~$Y$ соответственно.
Докажите, что $\angle XDF = \angle YDE$.

\item
Треугольник $ABC$ вписан в~окружность~$\omega$ с~центром~$O$.
Прямая~$AO$ пересекает $\omega$ вторично в~точке~$A'$.
Серединный перпендикуляр к~$OA'$ пересекает $BC$ в~точке~$P_A$.
Точки $P_B$ и~$P_C$ определяются аналогично.
Докажите, что
\\
\subproblem
точки $P_A$, $P_B$, $P_C$ коллинеарны;
\\
\subproblem
Расстояние от~точки~$O$ до~прямой $P_A P_B P_C$ равно $\frac{R}{2}$, где
$R$~--- радиус окружности~$\omega$.

\item
Точки $A$, $B$, $C$ и~$D$ лежат на~прямой~$\ell$ в~указанном порядке.
Дуги $\gamma_1$ и~$\gamma_2$, лежащие по~одну сторону от~$\ell$, проходят
через точки $A$ и~$B$.
Дуги $\gamma_3$ и~$\gamma_4$ проходят через $C$ и~$D$.
Оказалось, что $\gamma_1$ касается $\gamma_3$, а~$\gamma_2$ касается
$\gamma_4$.
Докажите, что общая внешняя касательная к~$\gamma_2$ и~$\gamma_3$ и~общая
внешняя касательная к~$\gamma_1$ и~$\gamma_4$
пересекаются на~прямой~$\ell$.

\item
Четырехугольник $ABCD$ вписан в~окружность~$\omega$.
Пусть $I_1$, $I_2$ и~$r_1$, $r_2$~--- центры и~радиусы вписанных
окружностей $ACD$ и~$ABC$ соответственно.
Предположим, что $r_1 = r_2$.
Окружность~$\omega'$ касается $AB$, $AD$ и~$\omega$ в~точке~$T$.
Касательные в~точках $A$, $T$ к~$\omega$ пересекаются в~точке~$K$.
Докажите, что $I_1$, $I_2$ и~$K$ лежат на~одной прямой.

\item
В~треугольнике $ABC$ проведены биссектриса~$AD$ и~высота~$AH$ из~вершины~$A$.
Серединный перпендикуляр к~$AD$ пересекает полукруги с~диаметрами $AB$ и~$AC$,
лежащие вне треугольника $ABC$ в~точках $X$ и~$Y$ соответственно.
Докажите, что четырехугольник $XYDH$ вписаный.

\end{problems}

