% $date: 2015-11-13
% $timetable:
%   g9r2:
%     2015-11-13:
%       1:
%   g9r1:
%     2015-11-13:
%       2:

\section*{Инверсия и поворот}

% $authors:
% - Фёдор Львович Бахарев

\begin{problems}

\item
\subproblem
Пусть при инверсии с~центром~$O$ точка~$A$ переходит в~$A'$, а~точка~$B$
переходит в~$B'$.
Докажите, что треугольники $OAB$ и~$OB'A'$ подобны.
\\
\subproblem
Докажите, что при инверсии с~центром~$O$ прямая~$\ell$, не~проходящая
через $O$, переходит в~окружность, проходящую через $O$, а~окружность,
проходящая через $O$ переходит в~прямую, не~проходящую через $O$.
\\
\subproblem
Докажите, что при инверсии с~центром $O$ окружность, не~проходящая через
$O$ переходит в~окружность, не~проходящую через $O$.

\item
\subproblem
Докажите, что при инверсии касающиеся окружности (прямая и~окружность)
переходят в~касающиеся окружности, или в~касающиеся окружность и~прямую,
или в~пару параллельных прямых.
\\
\subproblem
Докажите, что при инверсии сохраняется угол между окружностями (между
окружностью и~прямой, между прямыми).

\item
Четырехугольник $ABCD$ вписан в~окружность, $AB \cdot CD = AD \cdot BC$.
При инверсии с~центром в~точке~$A$ и~радиусом~1
точка~$B$ переходит в~$B'$, $C$ в~$C'$, а~$D$ в~$D'$.
%точка~$B$ переходит в~$B'$, $C$~--- в~$C'$ и~$D$~--- в~$D'$.
Докажите, что $B'C' = C'D'$.

\item
На~сторонах треугольника $ABC$ внешним образом построены правильные
треугольники $A_1 B C$, $A B_1 C$ и~$A B C_1$.
Пусть $P$ и~$Q$~--- середины отрезков $A_1 B_1$ и~$A_1 C_1$.
Докажите, что треугольник $APQ$ правильный.

\item
На~сторонах $AB$ и~$AC$ треугольника $ABC$ построены равнобедренные
треугольники $ADB$ ($AD = AB$) и~$ACE$ ($AC = AE$), причем угол $DAE$ равен
сумме углов $ABC$ и~$ACB$.
Докажите, что отрезок~$DE$ в~два раза длиннее медианы~$AM$ треугольника $ABC$.

\item
Две окружности пересекаются в~точках $A$ и~$B$.
Прямая, проходящая через точку~$A$, пересекает первую окружность в~точке~$C$,
а~вторую~--- в~точке~$D$.
Пусть $M$ и~$N$~--- середины дуг $BC$ и~$BD$, не~содержащих точку~$A$,
и~$K$~--- середина отрезка~$CD$.
Докажите, что угол $MKN$~--- прямой.

\item
Дана окружность и~точка~$P$ внутри нее, отличная от~центра.
Рассматриваются пары окружностей, касающиеся данной изнутри и~друг друга
в~точке~$P$.
Найдите геометрическое место точек пересечения общих внешних касательных к~этим
окружностям.

\item
Докажите, что две непересекающиеся окружности $S_1$ и~$S_2$ можно при помощи
инверсии перевести в~пару концентрических окружностей.

\end{problems}

