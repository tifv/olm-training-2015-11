% $date: 2015-11-14
% $timetable:
%   g9r2:
%     2015-11-14:
%       3:
%   g9r1:
%     2015-11-14:
%       2:

\section*{Повторительный разнобой}

% $authors:
% - Фёдор Львович Бахарев

\begin{problems}

\item
В~остроугольном треугольнике $ABC$ на~высоте $BK$ как на~диаметре построена
окружность~$S$, пересекающая стороны $AB$ и~$BC$ в~точках $E$ и~$F$
соответственно.
К~окружности~$S$ в~точках $E$ и~$F$ проведены касательные.
Докажите, что точка их пересечения лежит на~медиане треугольника,
проведенной из~вершины~$B$.

\item
В~окружности~$S$ проведены две параллельные хорды $AB$ и~$CD$.
Прямая, проведенная через $C$ и~середину $AB$, вторично пересекает окружность
в~точке~$E$.
Точка~$K$~--- середина отрезка~$DE$.
Докажите, что $\angle AKE = \angle BKE$.

\item
На~диагоналях $AC$ и~$BD$ вписанного четырехугольника $ABCD$ выбраны точки
$M$ и~$N$ соответственно такие, что $BN : DN = AM : CM$.
Оказалось, что $\angle BAD = \angle BMC$.
Докажите, что $\angle ADC = \angle BNA$.

\item
Внутри окружности зафиксирована точка~$A$, а~на~окружности взяты произвольные
точки $B$ и~$C$ так, что $\angle BAC = 90^{\circ}$.
Найдите геометрическое место середин хорд~$BC$.

\item
Внутри окружности зафиксирована точка~$A$, а~на~окружности взяты произвольные
точки $B$ и~$C$ так, что $\angle BAC = 90^{\circ}$.
Найдите геометрическое место точек пересечения касательных к~окружности
в~точках $B$ и~$C$.

\item
Никакие три из~четырех точек $A$, $B$, $C$, $D$ не~лежат на~одной прямой.
Докажите, что угол между описанными окружностями треугольников $ABC$ и~$ABD$
равен углу между описанными окружностями треугольников $ACD$ и~$BCD$.

\item
Ортоцентр~$H$ треугольника $ABC$ лежит на~вписанной в~треугольник окружности.
Докажите, что три окружности с~центрами $A$, $B$, $C$, проходящие через $H$,
имеют общую касательную.

\item
Окружность~$S_A$ проходит через точки $A$ и~$C$;
окружность~$S_B$ проходит через точки $B$ и~$C$;
центры обеих окружностей лежат на~прямой~$AB$.
Окружность~$S$ касается окружностей $S_A$ и~$S_B$, а~кроме того, она касается
отрезка~$AB$ в~точке~$C_1$.
Докажите, что $C C_1$~--- биссектриса треугольника $ABC$.

\end{problems}

