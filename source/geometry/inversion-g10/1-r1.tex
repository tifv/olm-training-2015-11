% $date: 2015-11-14
% $timetable:
%   g10r1:
%     2015-11-14:
%       2:

\section*{Инверсия-1}

% $authors:
% - Фёдор Константинович Нилов

\begin{problems}

\item
Докажите, что при инверсии
\\
\subproblem
окружность, проходящая через центр инверсии, переходит в~прямую;
\\
\subproblem
окружность, не~проходящая через центр инверсии, переходит в~окружность,
не~проходящую через центр инверсии.

\item
Точки $P'$ и~$Q'$~--- образы точек $P$ и~$Q$ при инверсии относительно
окружности с~центром~$O$ радиуса $R$, причем точки $P$ и~$Q$ отличны от~$O$.
Докажите, что
\[
    P'Q' = PQ \cdot \frac{R^2}{OP \cdot OQ}
\; . \]

\item
Докажите, что касающиеся окружности (окружность и~прямая) переходят при
инверсии в~касающиеся окружности или в~окружность и~прямую, или в~пару
параллельных прямых.

\item
Докажите, что при инверсии сохраняется угол между окружностями (между
окружностью и~прямой, между прямыми).

\item
Окружности $\omega_1$, $\omega_2$, $\omega_3$ и~$\omega_4$ таковы, что
$\omega_2$ и~$\omega_4$ касаются каждой из~окружностей $\omega_1$ и~$\omega_3$.
Докажите, что точки касания лежат на~одной окружности или прямой.

\item
Дана полуокружность~$\omega$ с~диаметром~$PQ$.
Окружность~$\alpha$ касается $\omega$ внутренним образом и~отрезка~$PQ$
в~точке~$C$.
Прямая~$l$ перпендикулрна $PQ$ и~касается $\alpha$.
Пусть она пересекает дугу~$\omega$ в~точке~$A$ и~отрезок~$PQ$ в~точке~$B$.
Докажите, что $AC$ делит угол $PAB$ пополам.

\item
Докажите \emph{формулу Эйлера:} $d^2 = R^2 - 2 R r$, где $d$~--- расстояние
между центрами вписанной и описанной окружностей, а $r$ и $R$~--- их радиусы.

\item \emph{Неравенство Птолемея.}\enspace
Даны произвольные 4~точки.
Докажите, что
\[
    AC \cdot BD
\leq
    AD \cdot BC + AB \cdot CD
\, . \]

\item
На~прямой~$l$ даны точки $A$, $B$ и~$C$ в~данном порядке.
Полуокружности $\alpha$ и~$\beta$ построены на~отрезках $AB$ и~$BC$ как
на~диаметрах по~одну сторону от~прямой~$l$.
Окружность~$\gamma$ касается полуокружности~$\alpha$, полуокружности~$\beta$
в~точке~$T$ и~перперпендикуляра к~прямой~$l$ в~точке~$C$.
Докажите, что $AT$ касается $\beta$.

\item
Дан треугольник $ABC$ и~точка~$P$ внутри такая, что
$\angle APB - \angle C = \angle APC - \angle B$.
Докажите, что биссектрисы углов $ABP$ и~$ACP$ пересекаются на~прямой~$AP$.

\item \emph{Теорема Фейербаха.}\enspace
Докажите, что окружность, проходящая через середины сторон неравнобедренного
треугольника касается его вписанной окружности.

\item
Вписанная окружность треугольника $ABC$ касается его сторон в~точках~$A_1$,
$B_1$ и~$C_1$.
Докажите, что ортоцентр треугольника $A_1 B_1 C_1$ лежит на~прямой, соединяющей
центры вписанной и~описанной окружностей треугольника $ABC$.

\item
Дан треугольник $ABC$.
Обозначим через $I$ центр вписанной окружности.
Пусть $A_1$, $B_1$ и~$C_1$~--- точки касания с~соответствующими сторонами.
Пусть $\omega_A$~-- меньшая из~двух окружностей окружностей, проходящих через
точку~$I$ и~вписанных в~угол~$A$.
Аналогично определим окружности $\omega_B$ и~$\omega_C$.
Обозначим через $A_2$ вторую точку пересечения окружностей $\omega_B$
и~$\omega_C$, отличную от~$I$.
Аналогично определим точки $B_2$ и~$C_2$.
Докажите, что центры описанных окружностей треугольников
\\
\subproblem $A I A_1$, $B I B_1$ и~$C I C_1$
\qquad
\subproblem $A I A_2$, $B I B_2$ и~$C I C_2$
\\
лежат на~одной прямой.

\end{problems}

