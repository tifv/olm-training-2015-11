% $date: 2015-11-16
% $timetable:
%   g10r1:
%     2015-11-16:
%       1:

\section*{Стереографическая проекция и~инверсия в~пространстве}

% $authors:
% - Фёдор Константинович Нилов

\begin{problems}

\item
Докажите, что при стереографической проекции окружности переходят в~окружности
или прямые.

\item
Докажите, что при стереографической проекции сохраняется угол между
окружностями.

\item
Дана точка вне сферы и~окружность, лежащая на~этой сфере.
Докажите, что вторые точки пересечения со~сферой прямых, соединяющих данную
точку с~точками данной окружности, лежат на~одной окружности.

\item
B основании четырехугольной пирамиды $SABCD$ лежит четырехугольник $ABCD$,
диагонали которого перпендикулярны и~пересекаются в~точке~$P$, и~$SP$ является
высотой пирамиды.
Докажите, что проекции точки~$P$ на~боковые грани пирамиды лежат на~одной
окружности.

\item
Сфера с~центром в~плоскости основания $ABC$ тетраэдра $SABC$ проходит через
вершины $A$, $B$ и~$C$ и~вторично пересекает ребра $SA$, $SB$ и~$SC$
в~точках $A_1$, $B_1$ и~$C_1$ соответственно.
Плоскости, касающиеся сферы в~точках $A_1$, $B_1$ и~$C_1$, пересекаются
в~точке~$O$.
Докажите, что $O$~--- центр сферы, описанной около тетраэдра $S A_1 B_1 C_1$.

\item
Четырехугольная пирамида $SABCD$ вписана в~сферу.
Из~вершин $A$, $B$, $C$, $D$ опущены
перпендикуляры $A A_1$, $B B_1$, $C C_1$, $D D_1$
на~прямые $SC$, $SD$, $SA$, $SB$ соответственно.
Оказалось, что точки $S$, $A_1$, $B_1$, $C_1$, $D_1$ различны и~лежат
на~одной сфере.
Докажите, что точки $A_1$, $B_1$, $C_1$, $D_1$ лежат в~одной плоскости.

\end{problems}

