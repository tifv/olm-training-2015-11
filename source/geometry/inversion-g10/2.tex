% $date: 2015-11-15
% $timetable:
%   g10r2:
%     2015-11-15:
%       1:
%   g10r1:
%     2015-11-15:
%       3:

\section*{Инверсия-2}

% $authors:
% - Фёдор Константинович Нилов

\begin{problems}

\item
Пусть $K$, $M$, $N$~--- произвольные точки на~окружности, $p$~--- серединный
перпендикуляр к~отрезку~$MN$.
Докажите, что прямые $KM$ и~$KN$ пересекают прямую~$p$ в~точках $A$ и~$B$,
инверсных относительно окружности.

\item
В~сегмент вписываются всевозможные пары касающихся окружностей.
Найдите множество их точек касания.

\item \emph{Задача Архимеда.}
Пусть точка~$C$ лежит на~отрезке~$AB$.
Построим в~одну сторону от~отрезка полуокружности
на~диаметрах $AB$, $BC$, $AC$.
Фигура, ограниченная данными дугами, называется \emph{арбелосом}.
Перпендикуляр~$MC$ к~отрезку~$AB$ делит арбелос на~две части.
Докажите, что радиусы окружностей, вписанных в~эти части арбелоса, равны между
собой.

\item \emph{Задача Паппа.}
Даны окружности $\alpha$, $\beta$ и~$\gamma$ с~диаметрами $AB$, $BC$, $AC$,
которые образуют арбелос.
Пусть $\omega_1$~--- окружность, вписанная в~арбелос,
окружность~$\omega_2$ касается окружностей $\alpha$, $\beta$ и~$\omega_1$,
окружность~$\omega_3$ касается окружностей $\alpha$, $\beta$ и~$\omega_2$,
\ldots,
окружность~$\omega_{n+1}$ касается окружностей $\alpha$, $\beta$ и~$\omega_n$.
Пусть $r_n$~--- радиус окружности~$\omega_n$, $d_n$~--- расстояние от~центра
окружности~$\omega_n$ до~прямой~$AB$.
Докажите, что тогда расстояние от~центра $k$-й окружности до~диаметра арбелоса
в~$2k$ раз больше ее радиуса.

\item
Дан треугольник $ABC$.
Окружность~$\gamma$ вписана в~угол $ABC$ и~касается описанной окружности $ABC$
в~точке~$P$.
Вневписанная в~угол~$B$ окружность касается стороны~$AC$ в~точке~$Q$.
Докажите, что $\angle ABP = \angle CBQ$.

\item
Углы $AOB$ и~$COD$ совмещаются поворотом так, что луч~$OA$ совмещается
с~лучом~$OC$, а~луч~$OB$~--- с~$OD$.
В~них вписаны окружности, пересекающиеся в~точках $E$ и~$F$.
Докажите, что углы $AOE$ и~$DOF$ равны.

\end{problems}

