% $date: 2015-11-12
% $timetable:
%   gX:
%     2015-11-12:
%       3:

\section*{Иранская геометрия}

% $authors:
% - Фёдор Львович Бахарев

\begin{problems}

\item
В~описанной окружности треугольника $ABC$ дан диаметр~$AA'$.
Проведем прямые $\ell'$ и~$\ell$ проходящие через точку~$A'$,
параллельные соответственно внутренней и~внешней биссектрисе угла~$A$
треугольника.
Пусть $\ell'$ пересекает $AB$ и~$BC$ в~точках $B_1$ и~$B_2$, а~$\ell$
пересекает $AC$ и~$BC$ в~точках $C_1$ и~$C_2$.
Докажите, что описанные окружности треугольников $ABC$, $C C_1 C_2$
и~$B B_1 B_2$ имеют общую точку.

\item
Треугольник $ABC$ равнобедренный ($AB = AC$).
Точки $P$ и~$Q$ внутри треугольника таковы, что $Q$ лежит внутри
угла $\angle PAC$ и~$\angle PAQ = \angle BAC / 2$.
Кроме того, $BP = PQ = CQ$.
Пусть $X = AP \cap BQ$ и~$Y = AQ \cap CP$.
Докажите, что четырехугольник $PQYX$ вписан.

\item
Различные точки $B$, $B'$, $C$, $C'$ лежат на~прямой~$\ell$.
$A$~--- точка вне прямой~$\ell$.
Прямая, проходящая через $B$ параллельно $AB'$, пересекает $AC$ в~точке~$E$,
а~прямая, проходящая через $C$ параллельно $AC'$, пересекает $AB$ в~точке~$F$.
Пусть $X$~--- точка пересечения описанных окружностей $\triangle ABC$
и~$\triangle AB'C'$ ($A \neq X$).
Докажите, что $EF \parallel AX$.

\item
Пусть $D$~--- точка на~стороне~$BC$ треугольника $ABC$.
Окружность~$\omega_1$ касается отрезков $AD$ и~$BD$ и~описанной окружности
треугольника $ABC$.
Окружность~$\omega_2$ касается отрезков $AD$ и~$CD$ и~описанной окружности
треугольника $ABC$.
Пусть $X$ и~$Y$~--- точки касания окружностей $\omega_1$ и~$\omega_2$ с~$BC$
соответственно, и~$M$~--- середина $XY$.
Пусть $T$~--- середина дуги~$BC$, не~содержащей точку~$A$.
Докажите, что если $I$~--- центр вписанной окружности треугольника
$ABC$, то~$TM$ проходит через середину $ID$.

\item
Пусть $X$ и~$Y$~--- две точки, лежащие на~продолжениях стороны~$BC$
треугольника $ABC$ такие, что $\angle XAY = 90^\circ$.
Пусть $H$~--- ортоцентр треугольника $ABC$.
Положим $X' = BH \cap AX$ и $Y' = CH \cap AY$.
Докажите, что описанная окружность треугольника $CYY'$, описанная окружность
треугольника $BXX'$ и~прямая~$X'Y'$ пересекаются в~одной точке.

\item
Пусть $I$~--- центр вписанной окружности треугольника $ABC$.
Перпендикуляр из~$I$ к~$AI$ пересекает $AB$ и~$AC$ в~точках $B'$ и~$C'$
соответственно.
Предположим, что $B''$ и~$C''$~--- точки на~лучах $BC$ and $CB$ соответственно,
такие что $B B'' = BA$ и~$C C'' = CA$.
Предположим, что вторая точка пересечения описанных окружностей
треугольников $A B' B''$ и~$A C' C''$ есть точка~$T$.
Докажите, что центр описанной окружности треугольника $AIT$ лежит на~$BC$.

\item
Точка~$D$ лежит на~стороне~$BC$ треугольника $ABC$.
$I$, $I_1$ и~$I_2$~--- центры вписанных окружностей $ABC$, $ABD$ и~$ACD$
соответственно.
$M \neq A$ и $N \neq A$~--- точки пересечения описанной окружности треугольника
$ABC$ с~описанными окружностями треугольников $I A I_1$ и~$I A I_2$
соответственно.
Докажите, что прямая~$MN$ проходит через постоянную точку, не~зависящую
от~выбора точки~$D$.

\item
Вписанная окружность неравнобедренного треугольника $ABC$ с~центром~$I$
касается стороны~$BC$ в~точке~$D$.
Пусть $X$~--- точка на~дуге~$BC$ описанной окружности треугольника $ABC$, такая
что если $E$ и~$F$ являются основаниями перпендикуляров из~$X$ на~$BI$ и~$CI$,
а~$M$~--- середина $EF$, то~$MB = MC$.
Докажите, что $\angle BAD = \angle CAX$.

\end{problems}

