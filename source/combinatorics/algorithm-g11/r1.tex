% $date: 2015-11-16
% $timetable:
%   g11r1:
%     2015-11-16:
%       1:

\section*{Алгоритмы}

% $authors:
% - Николай Евгеньевич Крохмаль

% $build$style[print]:
% - .
% - resize font: [9.75, 11.70]

\begin{problems}

%\item
%Клетки доски $8 \times 8$ раскрашены в~шахматном порядке.
%Разрешается менять местами любые две горизонтали или любые две вертикали.
%Можно~ли при помощи этих операций получить доску, вся левая половина которой
%окрашена в~черный цвет, а~правая половина~---  в~белый?

\item
На~плоскости нарисован квадрат и~невидимая точка, не~лежащая на~границе
квадрата.
За~один ход Вася может провести прямую и~спросить, по~какую сторону лежит точка
(если точка лежит на~прямой, он получает произвольный ответ).
За~какое наименьшее число вопросов он сможет узнать, лежит~ли точка внутри
квадрата?

\item
Двое игроков ставят крестики и~нолики на~бесконечной клетчатой бумаге, первый
крестик, второй~--- нолик, первый~--- крестик, второй~--- нолик и~т.~д.
Докажите, что первый может добиться, чтобы некоторые четыре крестика
образовывали квадрат.

\item
За~столом сидят $16$~джедаев.
Любознательный Чубакка хочет узнать, как их зовут.
Чубакка может выбрать произвольное подмножество джедаев и~попросить
мастера Йоду за~один имперский кредит перечислить все их имена.
К~сожалению, порядок, в~котором Йода перечисляет имена, может быть
произвольный.
Какого наименьшего количества имперских кредитов хватит Чубакке
(имена у~джедаев разные)?

\item
Стража ловит вора, забравшегося во~дворец к~султану.
Чтобы поймать вора, стражнику нужно оказаться с~ним в~одной комнате.
Дворец состоит из~$1000$ комнат, соединенных дверьми.
Планировка дворца такова, что из~комнаты в~соседнюю комнату нельзя пройти
иначе, как через соединяющую их (всегда единственную) дверь.
Могут~ли стражники при любой такой планировке составить план действий,
гарантирующий поимку вора, если их
\\
\subproblem 10;
\qquad
\subproblem 5;
\qquad
\subproblem 6.

\item
В~дом отдыха с~четырехразовым питанием на~$15$ дней приехала группа из~$60$
отдыхающих.
За~круглым обеденным столом $61$~место.
На~одном месте постоянно сидит директор дома отдыха.
Директор хочет сам познакомиться со~всеми отдыхающими и~перезнакомить их между
собой.
Для этого он хочет сажать отдыхающих каждый раз по-новому, чтобы ни~один из~них
не~сидел дважды на~одном и~том~же месте и~чтобы у~всех отдыхающих и~у~директора
каждый раз был новый сосед справа.
Как это сделать?

\item
На~плоскости поставлено $2015$ фишек, причем они не~стоят все на~одной прямой.
Можно переставлять любую фишку в~точку, симметричную ей относительно любой
другой фишки.
Докажите, что за~несколько таких операций можно добиться того, чтобы фишки
стояли в~вершинах некоторого выпуклого многоугольника.

\item
На~табло горят несколько лампочек.
Имеется несколько кнопок.
Нажатие на~кнопку меняет состояние лампочек, с~которыми она соединена.
Известно, что для любого набора лампочек найдется кнопка, соединенная
с~нечетным числом лампочек из~этого набора.
Докажите, что, нажимая на~кнопки, можно погасить все лампочки.

\item
Главная аудитория фирмы <<Рога и~копыта>> представляет собой квадратный зал
из~восьми рядов по~восемь мест.
$64$ сотрудника фирмы писали в~этой аудитории тест, в~котором было шесть
вопросов с~двумя вариантами ответа на~каждый.
Могло~ли так оказаться, что среди наборов ответов сотрудников нет одинаковых,
причем наборы ответов любых двух людей за~соседними столами совпали не~больше,
чем в~одном вопросе?
(Столы называются соседними, если они стоят рядом в~одном ряду или друг
за~другом в~соседних рядах.)

\end{problems}

