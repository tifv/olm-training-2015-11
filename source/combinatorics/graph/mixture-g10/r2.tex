% $date: 2015-11-17
% $timetable:
%   g10r2:
%     2015-11-17:
%       1:

\section*{Графы с условиями}

% $authors:
% - Владимир Алексеевич Брагин

\begin{problems}

\item
\subproblem
В~компании из~6 людей среди любых троих людей есть двое знакомых.
Докажите, что есть трое людей, попарно знакомых между собой.
\\
\subproblem
В~компании из~9 людей среди любых троих есть двое знакомых.
Докажите, что есть четверо людей, попарно знакомых между собой.

\item
В~стране 100 городов, некоторые пары из~них соединены дорогами.
Известно, что любые 4~города можно объехать по~маршруту из~трех дорог.
Какое наименьшее количество ребер может быть в~таком графе?

\item
На~турнир приехали 100 человек.
Известно, что среди любых 50 из~них есть человек, знакомый с~остальными 49.
Докажите, что можно найти 52~человека, любые два из~которых знакомы между
собой.

\item \emph{Задача исключена.}
В~компании из~2015 людей среди любых пяти есть трое попарно знакомых.
Для какого наибольшего $n$ можно утверждать, что среди этих людей
гарантированно есть $n$ попарно знакомых людей?

\item
В~графе со~100 вершинами нет треугольников, а~степень каждой вершины больше 40.
\\
\subproblem
Докажите, что в~этом графе нет циклов длины 5.
\\
\subproblem
Докажите, что этот граф двудольный.

\item
На~вечеринку пришло
\quad
\subproblem 19 гостей,
\quad
\subproblem 21 гость,
\\
причем среди любых трех из~них есть двое знакомых.
Докажите, что гости могут разбиться на~5 групп, в~каждой из~которых
все попарно знакомы.

\item
\subproblem
\label{combinatorics/graph/mixture-g10/r1:problem:6a}%
В~волейбольном однокруговом турнире на~$n$ команд оказалось, что для любых
$k$~команд найдется выигравшая у~них всех.
Докажите, что\enspace
$n \geq 2^{k+1} - 1$.
\\
\subproblem
Пусть оказалось, что для любых $k$~команд найдутся $l$~команд, выигравших
у~них.
Докажите, что\enspace
$n \geq 2^k (l + 1) - 1$.
\\
\subproblem
В~условиях пункта~\ref{combinatorics/graph/mixture-g10/r1:problem:6a}
докажите, что\enspace
$n \geq 2^{k-1} (k + 2) - 1$.
\\
\subproblem
Докажите, что ситуация, описанная
в~пункте~\ref{combinatorics/graph/mixture-g10/r1:problem:6a},
возможна для любого $k$. (При этом $n$ мы выбираем сами.)

\end{problems}

