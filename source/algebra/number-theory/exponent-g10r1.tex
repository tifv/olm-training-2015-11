% $date: 2015-11-13
% $timetable:
%   g10r1:
%     2015-11-13:
%       1:

\section*{Показательный разговор}

% $authors:
% - Владимир Викторович Трушков

\begingroup
    \providecommand\divides{\mathrel{\vert}}

\begin{problems}

\item
Пусть $p$~--- простое число, $d$~--- один из~делителей числа $(p - 1)$.
Выберем из~остатков $1, 2, \ldots, p - 1$ те, чей показатель по~модулю~$p$
равен $d$.
Чему равен остаток произведения выбранных чисел по~модулю~$p$?

\item
Докажите, что числа $1, 2, \ldots, p - 1$ можно расставить по~кругу так, что
для любых трех последовательных $a$, $b$, $c$ разность
$(b^2 - a c)$ будет делиться на~простое~$p$.

\item
Пусть $p$~--- нечетное простое число.
Докажите, что нечетные простые делители числа $(a^p - 1)$ делят $(a - 1)$ или
имеют вид $2 p x + 1$.

\item
Пусть $p$~--- нечетное простое число.
Докажите, что нечетные простые делители числа $a^p + 1$ делят $a + 1$ или имеют
вид $2 p x + 1$.

\item
Пусть $p$~--- нечетное простое число.
Докажите, что существует бесконечно много простых чисел вида $2 p x + 1$.

\item
Докажите, что все простые делители числа $2^{2^n} + 1$ имеют вид
$2^{n+1} x + 1$.

\item
Докажите, что $3$ является первообразным корнем по~модулю простого числа
$p = 2^n + 1$, где $n > 1$.

\item
Пусть $p$~--- простое число и~$S = 1^n + 2^n + \ldots + (p - 1)^n$.
Докажите, что
\[
    S
\equiv
    \begin{cases}
       -1, & \quad \text{если $n$ делится на~$(p - 1)$}
    \\
        0, & \quad \text{если $n$ не~делится на~$(p - 1)$}
    \end{cases}
\pmod{p}
\; . \]

\item
Докажите, что $2$ является первообразным корнем по~модулю~$3^n$.

\item
Найдите наименьшее $n$ такое, что $2^{2015} \divides 17^n - 1$.

\end{problems}

\definition
Многочлен $f(x) \in \mathbb{Z}_p [x]$ будем называть \emph{перестановочным,}
если значения $f(0), f(1), \ldots, f(p-1)$ попарно различны.

\begin{problems}

\item
Докажите, что по~модулю $101$ не~существует перестановочного многочлена
$100$-й степени.

\item
Докажите, что по~модулю~$p$ не~существует перестановочного многочлена
степени $d$, если $d \divides p - 1$.

\end{problems}

\endgroup % \def\divides

