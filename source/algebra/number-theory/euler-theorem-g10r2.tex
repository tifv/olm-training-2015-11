% $date: 2015-11-12
% $timetable:
%   g10r2:
%     2015-11-12:
%       3:

\section*{Вокруг малой теоремы Ферма и теоремы Эйлера}

% $authors:
% - Владимир Викторович Трушков

\begin{problems}

\item
Докажите, что $(m^5 \cdot n - m \cdot n^5)$ делится на~$30$ при любых целых $m$
и~$n$.

\item
Найдите все простые~$p$ такие, что $5^{p^2} + 1$ делится на~$p$.

\item
Докажите, что для любого простого $p > 5$ число $(p^4 - 1)$ кратно $240$.

\item
Докажите, что существует бесконечно много составных чисел вида $10^n + 3$.

\item
Докажите, что для любого натурального~$a$ число~$a^5$ оканчивается на~ту~же
цифру, что и~$a$.

\item
Докажите, что $k^{81}$ при делении на~$243$ дает остаток $1$ или $242$.

\item
Существует~ли такое натуральное~$n$, что число $2007 n$ заканчивается
на~$987654321$?

\item
Каковы три последние цифры числа $7^{999}$?

\item
Существует~ли такое натуральное~$k$, что сто последних цифр числа $1543^k$
совпадают с~ста последними цифрами числа $2007^k$?

\item
Докажите, что существует бесконечно много натуральных чисел~$n$, для которых
$2^n + n^2$ кратно $100$.

\item
Докажите, что для любого простого~$p$ существует бесконечно много чисел
$(2^n - n)$, кратных $p$.

\item
Найдите все натуральные $n > 1$, для которых $1^n + 2^n + \ldots + (n-1)^n$
кратна $n$.

\item
Найдите все числа, взаимно простые с~каждым из~чисел вида
$a_n = 2^n + 3^n + 6^n - 1$.

\end{problems}

