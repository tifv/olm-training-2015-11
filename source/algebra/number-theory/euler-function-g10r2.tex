% $date: 2015-11-13
% $timetable:
%   g10r2:
%     2015-11-13:
%       2:

\section*{Функция Эйлера}

% $authors:
% - Владимир Викторович Трушков

\begingroup
    \ifx\mathup\undefined
        \def\eulerphi{\upphi}%
    \else
        \def\eulerphi{\mathup{\phi}}%
    \fi
    \providecommand\divides{\mathrel{\vert}}
    \def\GCD{\operatorname{\text{НОД}}}
    \def\LCM{\operatorname{\text{НОК}}}

\begin{problems}

\itemy{0}
Докажите, что функция Эйлера мультипликативна, то есть
\[
    \eulerphi(m n) = \eulerphi(m) \cdot \eulerphi(n)
\quad
    \text{при $(m, n) = 1$}
\, . \]

\item
Докажите, что при $n > 2$ функция $\eulerphi(n)$ принимает только четные значения.

\end{problems}

\medskip

Для любых натуральных $m$ и~$n$ докажите

\begin{problems}

\item
\(
    \eulerphi(m) \cdot \eulerphi(n)
=
    \eulerphi \bigl( \LCM(m, n) \bigr)
    \cdot
    \eulerphi \bigl( \GCD(m,n) \bigr)
\).

\item
\(
    \eulerphi(m n)
=
    \eulerphi \bigl( \LCM(m, n) \bigr)
    \cdot
    \GCD(m,n)
\).

\item
\(
    \eulerphi(m) \cdot \eulerphi(n) \cdot
    \GCD(m,n)
=
    \eulerphi(m n)
    \cdot
    \eulerphi \bigl( \GCD(m,n) \bigr)
\).

\end{problems}

\medskip

\begin{problems}

\item
Пусть $m$ и~$n$~--- натуральные числа, причем $\GCD(m, n) > 1$.
Докажите неравенство $\eulerphi(m n) > \eulerphi(m) \cdot \eulerphi(n)$.

\end{problems}

\medskip

Решите уравнения:

\begin{problems}

\item
$\eulerphi(x) = 12$.
\qquad
\problem
$\eulerphi(x) = 18$.
\qquad
\problem
$x-\eulerphi(x) = 12$.
\qquad
\problem
$\eulerphi(x^2) = x^2 - x$.

\end{problems}

\medskip

\begin{problems}

\item
Докажите, что $\sum\limits_{d \divides n} \eulerphi(d) = n$.

\item
Докажите, что $n^m \equiv n^{m - \eulerphi(m)} \pmod{m}$ при натуральных
$m$, $n$ больших единицы.

\item
Окружность разделена $105$ точками на~$105$ равных частей.
Сколько можно построить различных замкнутых ломаных из~$105$ равных звеньев
с~вершинами в~этих точках?

\end{problems}

\endgroup % \def\eulerphi \def\divides \def\GCD \def\LCM

