% $date: 2015-11-11
% $timetable:
%   g10r1:
%     2015-11-11:
%       3:

\section*{Первообразные корни}

% $authors:
% - Владимир Викторович Трушков

\begingroup
    \ifx\mathup\undefined
        \def\eulerphi{\upphi}%
    \else
        \def\eulerphi{\mathup{\phi}}%
    \fi

\begin{problems}

\item
Докажите, что при $(a, m) = 1$ существует положительное $\delta$ с~условием
$a^{\delta} \equiv 1 \pmod{m}$.
Наименьшее из~таких чисел называется \emph{показателем,} которому $a$
принадлежит по~модулю~$m$.

\item
Докажите, что числа $a^{0}, a^{1}, \ldots, a^{\delta-1}$ по~модулю~$m$
несравнимы.

\item
Докажите, что $a^{s} \equiv a^{t} \pmod{m}$ ($s > t \geq 0$) тогда и~только
тогда, когда $s \equiv t \pmod{\delta}$.
В~частности, $a^{s} \equiv 1 \pmod{m}$ тогда и~только тогда, когда $s$ делится
на~$\delta$.

\item
Докажите, что $\delta$ являтся делителем $\eulerphi(m)$.

\item
Докажите, что
\(
    \eulerphi(d_1) + \eulerphi(d_2) + \ldots + \eulerphi(d_s)
=
    m
\), где суммирование ведется по~всем делителям числа~$m$.

\definition
Если $(a, m) = 1$ и~показатель~$a$ равен $\eulerphi(m)$, то~$a$ называется
\emph{первообразным корнем} по~модулю~$m$.

\item
Существует~ли первообразный корень по~модулю~$8$?

\item
Пусть по~модулю~$m$ существует первообразный корень.
Сколько тогда имеется первообразных корней?
Как их все найти?

\item
Докажите, что если $(p - 1)$ делит $d$, то~уравнение $x^d = 1$ имеет ровно
$d$~корней.

\item
Докажите, что по~модулю простого~$p$ существует первообразный корень.

\item
Сколько корней в $x \in \mathbb{Z}_p$ имеет уравнение $x^d = 1$?
Как решать это уравнение, если известен первообразный корень?

\item
Найдите первообразный корень по~модулю~$29$.

\item
Решите уравнение $1 + x + \ldots + x^6 \equiv 0 \pmod{29}$.

\end{problems}

%\item
%Многочлен $f(x) \in \mathbb{Z}_p [x]$ будем называть \emph{перестановочным,}
%если значения $f(0),\dots, f(p-1)$ попарно различны.
%\\
%\subproblem
%по~модулю 101 не~существует перестановочного многочлена 100 степени;
%\\
%\subproblem
%по~модулю~$p$ не~существует перестановочного многочлена степени $d$,
%если $d \divides (p - 1)$;

%\item
%Докажите, что числа $1, 2, \ldots, (p - 1)$  можно расставить по~кругу так,
%что для любых трех последовательных $a$, $b$, $c$ разность $(b^2 - a c)$ будет
%делиться на~простое~$p$.

\endgroup % \def\eulerphi

