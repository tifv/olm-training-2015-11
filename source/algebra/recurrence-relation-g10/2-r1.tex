% $date: 2015-11-19
% $timetable:
%   g10r1:
%     2015-11-19:
%       2:

% $caption: Реккуренты --- 2

\section*{Реккурентные последовательности --- 2}

% $authors:
% - Владимир Алексеевич Брагин

\begin{problems}

\item
Пусть $a_{n} = 7 a_{n-2} + 6 a_{n-3}$ для всех $n \geq 4$
и~$a_1 = 1$, $a_2 = 25$, $a_3 = 43$.
Напишите явно формулу общего члена последовательности.

\item
Садовник, привив черенок редкого растения, оставляет его расти два года,
а~затем ежегодно берет от~него по~6 черенков.
С~каждым новым черенком он поступает аналогично.
Сколько будет растений и~черенков на~$n$-м году роста первоначального растения?

\item
Рассмотрим соотношение $a_n = 9 a_{n-1} - 14 a_{n-2} + 8$.
\\
\subproblem
Придумайте любую последовательность, удовлетворяющую ему.
\\
\subproblem
Напишите формулу общего члена для последовательности
$a_1 = 22$, $a_2 = 112$, $a_n = 9 a_{n-1} - 14 a_{n-2} + 8$.

\item
Дана последовательность $a_n = 2^n + 3^n + 6^n - 1$.
Составьте для этой последовательности линейное реккурентное соотношение,
выражающее каждый член через
\\
\subproblem 4;
\quad
\subproblem 3
\quad
предыдущих.

\item
Последовательность $\{ x_n \}$ такова, что
$x_1 = 8$, $x_2 = 20$ и~$x_{n+2} = 4 x_{n+1} - 4 x_n$.
Выведите формулу общего члена для этой последовательности.

\item
Последовательность натуральных чисел $\{ a_{n} \}$ строится следующим образом:
$a_1 = 0$, $a_2 = 2$, $a_3 = 3$ и~$a_{n+3} = a_{n+1} + a_n$ для
натуральных~$n$.
Докажите, что для любого простого~$p$ число~$a_p$ делится на~$p$.

\item
Найдите все такие функции $f(x) \colon [0; +\infty) \to [0; +\infty)$, что для
любого $x \geq 0$ выполнено равенство $6 f(f(x)) + f(x) - x = 0$.

\item
По~вершинам пятиугольника прыгает кузнечик.
Каждый ход он прыгает в~соседнюю по~стороне вершину.
Найдите количество путей длины~$n$, возвращающих кузнечиика в~стартовую
вершину.

\end{problems}

