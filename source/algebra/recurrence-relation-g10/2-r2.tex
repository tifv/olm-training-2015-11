% $date: 2015-11-19
% $timetable:
%   g10r2:
%     2015-11-19:
%       1:

% $caption: Реккуренты --- 2

\section*{Реккурентные последовательности --- 2}

% $authors:
% - Владимир Алексеевич Брагин

\begin{problems}

\item
Рассмотрим соотношение $a_n = 6 a_{n-1} - 5 a_{n-2} - 4$.
\\
\subproblem
Придумайте любую последовательность, удовлетворяющую ему.
\\
\subproblem
Напишите формулу общего члена для последовательности
$a_1 = 3$, $a_2 = 8$, $a_n = 6 a_{n-1} - 5 a_{n-2} - 4$.

\item
Дана последовательность $a_n = 2^n + 3^n + 6^n - 1$.
Составьте для этой последовательности линейное реккурентное соотношение,
выражающее каждый член через
\\
\subproblem 4;
\quad
\subproblem 3
\quad
предыдущих.

\item
Последовательность $\{ x_n \}$ такова, что
$x_1 = 8$, $x_2 = 20$ и~$x_{n+2} = 4 x_{n+1} - 4 x_n$.
Выведите формулу общего члена для этой последовательности.

\item
Последовательность натуральных чисел $\{ a_{n} \}$ строится следующим образом:
$a_1 = 0$, $a_2 = 2$, $a_3 = 3$ и~$a_{n+3} = a_{n+1} + a_n$ для
натуральных~$n$.
Докажите, что для любого простого~$p$ число~$a_p$ делится на~$p$.

\item
По~вершинам пятиугольника прыгает кузнечик.
Каждый ход он прыгает в~соседнюю по~стороне вершину.
Найдите количество путей длины~$n$, возвращающих кузнечика в~стартовую
вершину.

\end{problems}

%\medskip

\rule[0.5ex]{\textwidth}{0.5pt}

%\medskip

\begin{problems}

\item
Последовательность $\{ a_{n} \}$ задана как
$a_1 = a_2 = 1$ и~$a_{n+2} = a_{n+1} + \frac{a_{n}}{3^n}$ при $n \geq 1$.
Докажите, что $a_n < 2$ при $n \geq 1$.

\item
Последовательность $\{ a_{n} \}$ такова, что
$a_1 = 1$, $a_{n+1} = a_n + \frac{1}{a_{n}}$.
Докажите, что $a_{100} > 14$.

\item
Последовательность $\{ x_n \}$ задается следующим образом:
\[
    x_1 = \frac{1}{2}
\quad\text{и}\quad
    x_{n+1}
=
    1 - x_{1} \cdot x_{2} \cdot \ldots \cdot x_{n}
\quad\text{при $n \geq 1$.}
\]
Докажите, что
\\
\subproblem $0{,}99 < x_{100}$;
\qquad
\subproblem $x_{100} < 0{,}991$.

\end{problems}

