% $date: 2015-11-18
% $timetable:
%   g10r2:
%     2015-11-18:
%       1:

\section*{Числа Фибоначчи и рекуррентные последовательности}

% $authors:
% - Владимир Алексеевич Брагин

\emph{Если кто не~знает.}
Назовем \emph{числами Фибоначчи} последовательность, заданную следующим
образом:
\[
    F_1 = F_2 = 1
\quad\text{и}\quad
    F_{n+2} = F_{n+1} + F_{n}
\, . \]

\begin{problems}

\item
Для каких $a$ и~$b$ для всех $n \in \mathbb{N}$ верно, что\enspace
$F_{n+5} = aF_{n+2} + b F_n$?

\item
Чему равна сумма\enspace
$F_1 + F_2 + \ldots + F_n$?

\item
Может~ли сумма семи последовательных чисел Фибоначчи быть числом Фибоначчи?

\item
Докажите, что $F_{k}$ и~$F_{k+1}$ взаимно простые.

\item
\subproblem
Докажите, что\enspace
\(
    F_{n+m}
=
    F_{n-1} F_m + F_n F_{m+1}
=
    F_{n+1} F_{m+1} - F_{n-1} F_{m-1}
\).
\\
\subproblem
Докажите, что\enspace
$(F_{m}, F_{n}) = F_{(m, n)}$.

\item
Докажите, что каждое число единственным образом можно представить в~виде суммы
чисел Фибоначчи так, чтобы в~ней одновременно не~было соседних.

\end{problems}

Произвольной \textbf{линейной рекуррентной последовательностью} назовем
последовательность $\{ a_i \}_{i=1}^\infty$, в~которой для фиксированных
$c_1, c_2, \ldots, c_k$ для любого $n \geq k + 1$ выполнено равенство
\[
    a_{n} = c_1 a_{n-1} + c_2 a_{n-2} + \ldots + c_k a_{n-k}
\, . \]

Определим сумму последовательностей $\{ a_i \} + \{ b_i \}$ как почленную сумму
(то~есть $\{ a_i \} + \{ b_i \} = \{ a_i + b_i \}$).
Определим $\lambda \cdot \{ a_i \} = \{ \lambda \cdot a_i \}$.

\begin{problems}

\item
Зафиксируем коэффициенты, определяющие последовательность.
Докажите, что если $A$ и~$B$~--- последовательности, удовлетворяющие
соотношению с~этими коэффицентами, то~последовательность
$\lambda_1 A + \lambda_2 B$ тоже удовлетворяет.

\item
Рассмотрим $M$~--- множество последовательностей, удовлетворяющих условиям
$a_{n} = 5 a_{n-1} - 6 a_{n-2}$.
\\
\subproblem
Какие знаменатели могут быть у~геометрических прогрессий, лежащих в~$M$?
\\
\subproblem
Пусть $\{ a_i \} \in M$, а~$a_1 = 7$, $a_2 = 17$.
Найдите формулу общего члена такой последовательности.

\item
Найдите формулу общего члена для чисел Фибоначчи.

\item
Пусть $a_{n} = 6 a_{n-1} - 5 a_{n-2}$ для всех $n \geq 3$
и~$a_1 = 2$, $a_2 = 6$.
Напишите явно формулу общего члена последовательности.

\item
Садовник, привив черенок редкого растения, оставляет его
расти два года, а~затем ежегодно берет от~него по~6 черенков.
С~каждым новым черенком он поступает аналогично.
Сколько будет растений и~черенков на~$n$-м году роста первоначального растения?

\item
Какому линейному рекуррентному соотношению удовлетворяют квадраты чисел
Фибоначчи?

\item
От~прямоугольника отрезают квадраты, сторона которого равна меньшей стороне
прямоугольника.
С~оставшимся прямоугольников проделывают то~же самое.
Оказалось, что так можно делать сколь угодно долго и~квадраты всегда будут
разными.
Чему равно отношение сторон прямоугольника?

\end{problems}

